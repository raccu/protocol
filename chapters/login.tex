\section{Login}
In this section we define a flow for logging in at an independent endpoint using an Auth
issued attestation, as well as the process of authenticating the user by the Auth. 

    \subsection{Flow}
    The login flow is given in Figure ~\ref{fig:login} and defined as follows:
        \begin{enumerate}
            \item A user MAY try to access an unauthorized resource on the Server via the Browser.
            \item Server MUST notify the user that he/she needs to authenticate and SHALL refer the Auth he/she 
                  can use to do so.
            \item Upon receiving the information about the Auth, the Browser MUST fetch an authentication challenge
                  from the Auth. It is REQUIRED for the challenge to be combined with the domain name and
                  displayed as a QR code for the user to scan it with the Mobile.
            \item When the user scans an aforementioned QR code with the Mobile, the Mobile MUST authenticate the 
                  user locally by using a built-in biometry. Upon authenticating, the Mobile will sign the 
                  challenge, domain name, and user's email. Afterwards, the Mobile MUST send those parameters 
                  alongside the signature to the Auth.
            \item Auth SHOULD notify the Browser that the authentication is successful and that the attestation is
                  ready.
            \item Browser MUST request the attestation from the Auth by providing a challenge that the Auth issued 
                  as a reference to the authentication session.
            \item Browser MUST submit the attestation to the Server as a proof of user's identity, and thus finishing 
                  the authentication flow.
            \item Server SHOULD issue a new authentication and authorization token to the Browser in order to complete
                  the login process.
        \end{enumerate}

        \begin{figure}[H]
    \centering
    \begin{sequencediagram}
        
        \newinst{A}{Mobile}{}
        \newinst[2]{B}{Browser}{}
        \newinst[2]{C}{Auth}{}
        \newinst[2]{D}{Server}{}

        \tiny
        \begin{call}{B}{GET /protected/resource/1}{D}{401 Unauthorized}\end{call}{C}
        \begin{call}{B}{GET /auth}{C}{200 OK {(challenge, domain)}}\end{call}{B}
        \mess{B}{QR{(challenge, domain)}}{A}
        \begin{call}{A}{POST /attestation {(challenge, email, Sign{(challenge \textbar\textbar email \textbar\textbar domain)})}}{C}{200 OK}\end{call}{A}
        \mess{C}{Authentication successful}{B}
        \begin{call}{B}{GET /attestation/\{challenge\}}{C}{200 OK {(attestation)}}\end{call}{B}
        \begin{call}{B}{POST /login {(attestation)}}{D}{200 OK {(Custom server-defined authorization token)}}\end{call}{B}

    \end{sequencediagram}
    \caption{Login protocol flow}
    \label{fig:login}
\end{figure}

    \subsection{Requirements}
    The following requirements are made for a login:
        \begin{enumerate}
            \item Provided email MUST exist in the Auth's store.\\
            \textit{Argument:} It is trivial that the Auth cannot authenticate a user it does not have any record
                               of.         

            \item A challenge that the Auth generates MUST rely on a cryptographically secure pseudo-random number 
                  generator (CSPRNG) and MUST NOT be predictable. The challenge SHALL NOT be reused.\\
            \textit{Argument:} If the challenge can be predicted or guessed, an unauthorized party might fetch the
                               attestation or perform an attack from Man-in-the-Middle (MITM) vector of attacks, for
                               instance a relay attack. On the other hand, if the challenge is reused, the system 
                               becomes susceptible to the replay attack. Due to that, the challenge should be 
                               viewed as a nonce.

            \item Mobile MUST NOT expose private key directly.\\
            \textit{Argument:} Since the digital signature is a basis of authentication, an utmost respect to 
                               protecting the confidentiality of the private key should be implemented.

            \item Mobile MUST always require a biometric authentication to allow access to digital signing 
                  capability.\\        
            \textit{Argument:} This is to avoid phishing attacks or other illegal attempts to get the user's 
                               digital signature without his/hers explicit approval.

            \item When a user is prompted to authenticate a scanned QR code, the Mobile MUST disclose the domain that 
                  requires the authentication and is located inside the QR code itself.\\
            \textit{Argument:} This is to stop phishing attacks and also provide the feed-forward to the user.

            \item An attestation SHOULD NOT be permanently stored in the Browser or on the Server but rather deleted once
                  its validity has been proved and the process of authentication is finished. It is RECOMMENDED to handle 
                  the attestation with an utmost care for its confidentiality.\\
            \textit{Argument:} Since the attestation proves the holder's identity, an attacker that obtains a valid
                               attestation might pose as the user identified by the attestation.

            \item An attestation MUST be checked for the time of issuing and \textit{old} attestations MUST NOT be
                  accepted. It is left up to the implementer to define \textit{old}, but it is RECOMMENDED to
                  decline an attestation that was issued over three minutes ago.\\
            \textit{Argument:} Since the attestation proves the holder's identity, an attacker that obtains a valid
                               attestation might pose as the user identified by the attestation. Given that the login
                               flow is not particularly long or dependant upon data entry, a three minute time-to-live 
                               (TTL) period is gracious enough to let the network exchanges happen.

            \item The communication between the Browser and Auth, the Mobile and Auth, and the Browser and Server MUST 
                  rely on a secure channel that provides confidentiality and verifies integrity. It is RECOMMENDED to use
                  up-to-date version of TLS protocol with a well-tested and computationally secure cipher suite.\\      
            \textit{Argument:} The exchanged information should remain confidential given that the challenge is used to
                               identify an authentication session, and later on an attestation is used to authenticate the
                               user.
        \end{enumerate}

    \subsection{Token}
    \lipsum[1]
        \begin{figure}[H]
            \centering
            \begin{tikzpicture}
                \begin{class}[fill=white, drop shadow, draw=black]{Auth provider token}{0 ,0}
                    \attribute{user-email : string}
                    \attribute{user-name : string}
                    \attribute{issued-at : timestamp}
                    \attribute{target-domain : string}
                    \attribute{issuer : string}
                    \attribute{signature : hex}
                \end{class}
            \end{tikzpicture}
            \caption{Fields included into authentication token}
        \end{figure}
