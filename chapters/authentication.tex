\section{Authentication}
In this section, a flow for authenticating the user by Auth and its use in logging in at an independent endpoint
is defined.
\medskip

The authentication flow is given in Figure~\ref{fig:authentication} on page~\pageref{fig:authentication} and 
successful authentication is defined as follows:
    \begin{enumerate}
        \item Browser MUST fetch an authentication challenge from the Auth. It is REQUIRED for the challenge 
                to be combined with the domain name and displayed as a QR code for the user to scan it with 
                the Mobile.
        \item When the user scans an aforementioned QR code with the Mobile, the Mobile MUST authenticate the 
                user locally using a built-in biometry while disclosing the domain that requires the authentication.
                Upon authenticating, the Mobile will send the challenge, domain name, and user's email alongside the
                signature of those parameters to the Auth.
        \item Auth SHOULD notify the Browser that the authentication is successful and that the attestation is
                ready.
        \item Browser MUST request the attestation from the Auth by providing a challenge that the Auth issued 
                as a reference to the authentication session.
    \end{enumerate}
    \begin{figure}[H]
    \centering
    \begin{sequencediagram}
        
        \newinst{A}{Mobile}{}
        \newinst[2]{B}{Browser}{}
        \newinst[2]{C}{Auth}{}

        \tiny
        \begin{call}{B}{GET /auth}{C}{200 OK {(challenge)}}\end{call}{B}
        \mess{B}{QR{(challenge, domain)}}{A}
        \begin{call}{A}{POST /attestation {(challenge, email, domain, Sign{(challenge\textbar\textbar email\textbar\textbar domain)})}}{C}{200 OK}\end{call}{A}
        \mess{C}{Authentication successful}{B}
        \begin{call}{B}{GET /attestation/\{challenge\}}{C}{200 OK {(attestation)}}\end{call}{B}
            
    \end{sequencediagram}
    \caption{Authentication protocol flow.}
    \label{fig:authentication}
\end{figure}
If the login fails for any reason related to the authentication as described by the protocol, an attempt to 
authenticate MUST be discarded. All of the following reasons are considered to be an authentication error:
    \begin{itemize}
        \item An email is not associated with any account.
        \item Challenge does not exist.
        \item The signature cannot be verified by an associated public key.
        \item The signature is invalid.
    \end{itemize}

    \subsection{Attestation structure}
    An attestation that Auth issues to the user upon authenticating has a structure defined in a Figure~\ref{fig:attestation}
    on page~\pageref{fig:attestation}.
    \begin{figure}[H]
    \centering
    \begin{tikzpicture}
        \begin{class}[fill=white, drop shadow, draw=black]{Attestation}{0 ,0}
            \attribute{version : string}
            \attribute{iat : timestamp}
            \attribute{email : string}
            \attribute{claim : string}
            \attribute{domain : string}            
            \attribute{issuer : string}
            \attribute{signature : hex}                
        \end{class}
    \end{tikzpicture}
    \caption{Structure of an attestation}
    \label{fig:attestation}
\end{figure}

    Within the structure, a \textit{version} represents a version of the protocol under which an attestation was issued, 
    while \textit{iat} stands for \textit{issued at} and represents a moment at which an attestation was issued. The 
    \textit{domain} contains a domain name of the web service or application that requested the authentication, while 
    \textit{issuer} contains a domain name of the Auth instance. When it comes to the user information, the \textit{email}
    field contains the user's email and a \textit{claim} field signifies a type of registration process that the user 
    underwent upon account creation. In the first version of the protocol, only email verification claim is available, but 
    this field was added with future changes in mind. Lastly, the \textit{signature} field contains a digital signature of 
    all of the other fields, signed by the issuer with their private key that matches their certified public key. 

    \subsection{Login based on the authentication flow}
    When a user tries to connect to the Server utilizing the Raccu protocol, a flow shown on the Figure~\ref{fig:login} 
    on page~\pageref{fig:login} is RECOMMENDED.
    \begin{figure}[H]
    \centering
    \begin{sequencediagram}
        
        \newinst{A}{Mobile}{}
        \newinst[2]{B}{Browser}{}
        \newinst[2]{C}{Auth}{}
        \newinst[2]{D}{Server}{}

        \tiny
        \begin{call}{B}{GET /protected/resource/1}{D}{401 Unauthorized}\end{call}{C}
        \begin{call}{B}{GET /auth}{C}{200 OK {(challenge, domain)}}\end{call}{B}
        \mess{B}{QR{(challenge, domain)}}{A}
        \begin{call}{A}{POST /attestation {(challenge, email, Sign{(challenge \textbar\textbar email \textbar\textbar domain)})}}{C}{200 OK}\end{call}{A}
        \mess{C}{Authentication successful}{B}
        \begin{call}{B}{GET /attestation/\{challenge\}}{C}{200 OK {(attestation)}}\end{call}{B}
        \begin{call}{B}{POST /login {(attestation)}}{D}{200 OK {(Custom server-defined authorization token)}}\end{call}{B}

    \end{sequencediagram}
    \caption{Login protocol flow}
    \label{fig:login}
\end{figure}
    If the login is successful, the Server SHOULD communicate with the Browser in a manner of its implementation. Both 
    authentication and authorization from that point forward to the session expiry, as defined by the Server, MUST be 
    within the Server's scope. The attestation used to initially authenticate the user MUST be deleted by both the Browser
    and the Server.       
