\section{Register}
Registration is an act of sharing the proof of identity between the user and the Auth. Even though the current 
version of the protocol defines only email-based authentication, the protocol has been designed with 
extensibility in mind, meaning that other types of verification claims can be added, if needed.

      \subsection{Register via email}
      The flow of registering an account via email is shown in Figure~\ref{fig:registerViaEmail} and successful 
      registration flow is defined as follows:
      \begin{enumerate}
            \item A user MUST provide a valid email.
            \item When Auth receives a valid registration request, email containing the verification code MUST 
                  be dispatched to the Email and a token MUST be returned to the Mobile.
            \item Upon receiving the verification code and entering it into the Mobile, the Mobile MUST provide 
                  a method to generate, store, and retrieve public key and an interface for signing an arbitrary 
                  message with the matching private key.
            \item Mobile MUST authenticate the user locally using a built-in biometry. Upon authenticating, 
                  the Mobile MUST finish an account creation by submitting a token, email, generated public key, 
                  and a digital signature of all the previous parameters, with a REQUIRED addition of the 
                  verification code, to the Auth.
      \end{enumerate}
      \begin{figure}[H]
    \centering
    \begin{sequencediagram}

        \newinst{A}{Email}{}
        \newinst[3]{B}{Mobile}{}
        \newinst[3]{C}{Auth}{}
        
        \tiny
        \begin{call}{B}{POST /verify/email {(email)}}{C}{200 OK {(token)}}\end{call}{B}
        \mess{C}{Send an email {(verification code)}}{A}
        \mess{A}{verification code}{B}
        \begin{call}{B}{POST /register/email {(token, email, PU\textsubscript{k}, Sign{(token\textbar\textbar email\textbar\textbar PU\textsubscript{k}\textbar\textbar verification code)})}}{C}{200 OK}\end{call}{B}

    \end{sequencediagram}
    \caption{Register via email protocol flow.}
    \label{fig:registerViaEmail}
\end{figure}       
      If the registration via email fails for any reason related to the semantics of the protocol, an attempt to
      register MUST be discarded and all the related data should be deleted. All the following reasons are 
      considered to be a semantic of the protocol:
      \begin{itemize}
            \item Email is already associated with another account.
            \item Token does not exist.
            \item Token is not associated with a given email.
            \item Verification code entered into Mobile does not match the expected verification code.
            \item The signature cannot be verified by a given public key.
            \item The signature is invalid.
      \end{itemize}
