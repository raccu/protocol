\section{Register}
 In the current version of the protocol, only an email-based registration is described in details, although 	
other types of verification claims can be easily added, if needed.

    \subsection{Register via email}
    The flow of registering an account via email (i.e. using an email verification claim) is shown in 
    Figure ~\ref{fig:registerViaEmail} and defined as follows:
    \begin{enumerate}
        \item A user MUST provide a valid name and a valid email.
        \item When Auth receives a valid registration request, email containing the verification code MUST 
                be dispatched to the provided email address.
        \item Upon receiving the verification code and entering it into the Mobile, the Mobile MUST provide 
                a method to generate, store, and retrieve public key and an interface for signing an arbitrary 
                message with the matching private key.
        \item Mobile MUST finish an account creation by submitting a name, email, generated public key, and 
                a digital signature of all of the previous parameters, with a REQUIRED addition of the 
                verification code, to the Auth.
    \end{enumerate}
  
    \begin{figure}[H]
    \centering
    \begin{sequencediagram}

        \newinst{A}{Email}{}
        \newinst[3]{B}{Mobile}{}
        \newinst[3]{C}{Auth}{}
        
        \tiny
        \begin{call}{B}{POST /verify/email {(email)}}{C}{200 OK {(token)}}\end{call}{B}
        \mess{C}{Send an email {(verification code)}}{A}
        \mess{A}{verification code}{B}
        \begin{call}{B}{POST /register/email {(token, email, PU\textsubscript{k}, Sign{(token\textbar\textbar email\textbar\textbar PU\textsubscript{k}\textbar\textbar verification code)})}}{C}{200 OK}\end{call}{B}

    \end{sequencediagram}
    \caption{Register via email protocol flow.}
    \label{fig:registerViaEmail}
\end{figure}

        \subsubsection{Requirements}
        The following requirements are made for a registration via email:
            \begin{enumerate}
                \item Provided email MUST NOT be associated with another account on Auth.\\
                \textit{Argument:} This is to prove that one user can only have a single account.

                \item Verification code MUST be at least 4 characters long and short lived, and MUST NOT be 
                    predictable.\\        
                \textit{Argument:} Verification code should posses a certain entropy and a level of randomness, 
                                just as well as a limited lifetime in order to reduce the likelihood of a 
                                successful brute-force attack.

                \item Verification code MUST NOT be revealed to third parties or sent over the network, except when
                    it's sent to the Email by Auth.\\        
                \textit{Argument:} Verification code represents a shared secret between the entities and revealing 
                                it to other parties or sending it over a (potentially insecure) network poses 
                                a confidentiality breach risk. 

                \item It is RECOMMENDED to reset verification code after a several unsuccessful verification 
                    code entries.\\        
                \textit{Argument:} This is to reduce a chance of a successful brute-force attack.

                \item Mobile MUST NOT expose private key directly.\\
                \textit{Argument:} Since the digital signature is a basis of authentication, an utmost respect to 
                                protecting the confidentiality of the private key should be implemented.

                \item Mobile MUST always require a biometric authentication to allow access to digital signing 
                    capability.\\        
                \textit{Argument:} This is to avoid phishing attacks or other illegal attempts to get the user's 
                                digital signature without his/hers explicit approval.

                \item Mobile MUST store generated keys in a trusted execution environment (TEE).\\        
                \textit{Argument:} This is to avoid a root attack or other system-level attacks that target storage 
                                memory or execution environment.

                \item The communication between a Mobile and Auth MUST rely on a secure channel that provides 
                    confidentiality and verifies integrity. It is RECOMMENDED to use up-to-date version of 
                    Transport Level Security (TLS) protocol with a well-tested and computationally secure 
                    cipher suite.\\      
                \textit{Argument:} Mobile and Auth exchange user's personal information and transmitting them over a 
                                public channel would pose a privacy concern, as well as create new vectors of attack, 
                                namely social engineering, phishing, and targeted brute-force attacks.
            \end{enumerate}
        
        \subsubsection{Successful registration}
        If the registration via email is successful, the Mobile MUST remember user's name, email, public key, and 
        have access to the procedure that signs an arbitrary message with the matching private key. The Auth MUST 
        remember user's name, email, and public key.

        \subsubsection{Unsuccessful registration}
        If the registration via email fails for any reason related to the semantics of the protocol, an attempt to
        register MUST be discarded and all of the related data should be deleted. It is RECOMMENDED to log an attempt
        in order to screen a source of multiple invalid requests. Any of the following reasons is considered to be
        a semantic of the protocol:
            \begin{itemize}
                \item Email is already associated with another account.
                \item Verification code entered into Mobile does not match an expected verification code.
                \item The signature cannot be verified with a given public key.
                \item The signature is invalid.
            \end{itemize}
        If the registration fails for any other reason (e.g. network outage), an appropriate message SHOULD be
        dispatched or left to the other protocols for handling, should they offer such capability.
