\section{Register}
Registration is an act of sharing the proof of identity between the user and Auth instance. The proof of identity
can be in a spectar of forms, including high confidence national issued documents and low confidence emails and 
phone numbers. Even though current version of Raccu protocol defines only email based authentication, the protocol
has been designed with extensibility in mind, meaning that other types of verification claims can be added with
little extra effort.

      \subsection{Register via email}
      The flow of registering an account via email (i.e. using an email verification claim) is shown in 
      Figure~\ref{fig:registerViaEmail} on page~\pageref{fig:registerViaEmail} and is defined as follows:
      \begin{enumerate}
            \item A user MUST provide a valid name and a valid email.
            \item When Auth receives a valid registration request, email containing the verification code MUST 
                  be dispatched to the provided email address.
            \item Upon receiving the verification code and entering it into the Mobile, the Mobile MUST provide 
                  a method to generate, store, and retrieve public key and an interface for signing an arbitrary 
                  message with the matching private key.
            \item Mobile MUST authenticate the user locally using a built-in biometry. Upon authenticating, 
                  the Mobile MUST finish an account creation by submitting a name, email, generated public key, 
                  and a digital signature of all of the previous parameters, with a REQUIRED addition of the 
                  verification code, to the Auth.
      \end{enumerate}
      \begin{figure}[H]
    \centering
    \begin{sequencediagram}

        \newinst{A}{Email}{}
        \newinst[3]{B}{Mobile}{}
        \newinst[3]{C}{Auth}{}
        
        \tiny
        \begin{call}{B}{POST /verify/email {(email)}}{C}{200 OK {(token)}}\end{call}{B}
        \mess{C}{Send an email {(verification code)}}{A}
        \mess{A}{verification code}{B}
        \begin{call}{B}{POST /register/email {(token, email, PU\textsubscript{k}, Sign{(token\textbar\textbar email\textbar\textbar PU\textsubscript{k}\textbar\textbar verification code)})}}{C}{200 OK}\end{call}{B}

    \end{sequencediagram}
    \caption{Register via email protocol flow.}
    \label{fig:registerViaEmail}
\end{figure}           
      
            \subsubsection{Successful registration}
            If the registration via email is successful, the Mobile MUST remember user's name, email, public key, and 
            have access to the procedure that signs an arbitrary message with the matching private key. Auth MUST 
            remember user's name, email, and public key.

            \subsubsection{Unsuccessful registration}
            If the registration via email fails for any reason related to the semantics of the protocol, an attempt to
            register MUST be discarded and all of the related data should be deleted. All of the following reasons are 
            considered to be a semantic of the protocol:
            \begin{itemize}
                  \item Email is already associated with another account.
                  \item Verification code entered into Mobile does not match the expected verification code.
                  \item The signature cannot be verified by a given public key.
                  \item The signature is invalid.
            \end{itemize}
            If the registration fails for any other reason (e.g. network outage), an appropriate message SHOULD be
            dispatched or left to other protocols for handling, should they offer such capability.
