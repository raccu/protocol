\section{Definitions}
Throughout the specification a following list of terms is widely used. In this section we define their meaning 
and what assumptions do we make on them.

\medskip
\textbf{Attestation}: Digitally signed claim in a form of a token that confirms holder's identity. Issued by 
an Auth instance in a format defined in this specification and a separate Raccu Auth Provider API Specification. 
A secure storing of the attestation by its holder is implied.

\medskip
\textbf{Email}: User's email client that allows him to gain read access rights on the email address he/she 
provided. We assume that the user is the only person with the access to this entity.

\medskip
\textbf{Mobile}: A mobile application developed according to this specification and a seperate Raccu Mobile 
Application Specification that communicates with the corresponding Auth instance. Access to the fingerprint 
reader and an enabled biometric authentication on the host smartphone is implied.

\medskip
\textbf{Web}: A client-side component in charge of rendering QR codes and dispatching appropriate messages 
as specified by this specification and a separate Raccu Client Component Specification.

\medskip
\textbf{Auth}: Central authentication provider that serves as a secure token store (STS) and conforms to 
this specification and a separate Raccu Auth Provider API Specification. Valid and trusted certificate 
issued as a part of PKI for this entity is implied.

\medskip
\textbf{Server}: An end web service or application that consumes an attestation issued by an Auth instance. 
Authorization service within the Server is assumed. Valid and trusted certificate issued as a part of PKI 
for this entity is implied.

\medskip
The key words "MUST", "MUST NOT", "REQUIRED", "SHALL", "SHALL NOT", "SHOULD", "SHOULD NOT", "RECOMMENDED", 
"MAY", and "OPTIONAL" in this specification are to be interpreted as described in 
\href{https://tools.ietf.org/html/rfc2119}{RFC 2119}.