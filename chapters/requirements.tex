\section{Requirements}
In this section, a list of requirements and arguments for their inclusion is given, as well as a list of flows 
that depend upon implementation of each requirement:

    \subsection{Verification codes and challenges}
        \begin{enumerate}
                \item Verification code MUST be at least six characters long, contain a character space of at least ten 
                    characters, and be short lived while it MUST NOT be predictable.\\     
                \textit{Argument:} Verification code should posses a certain entropy and a level of randomness, 
                                just as well as a limited lifetime in order to reduce the likelihood of a 
                                successful brute-force attack.\\
                \textit{Dependant flows:} Register via email, reset credentials, delete account.

                \item Verification code MUST NOT be revealed to third parties or sent over the network, except when
                    it's sent to the Email by Auth.\\        
                \textit{Argument:} Verification code represents a shared secret between the entities and revealing 
                                it to other parties or sending it over a (potentially insecure) network poses 
                                a confidentiality breach risk.\\
                \textit{Dependant flows:} Register via email, reset credentials, delete account.

                \item Verification code MUST be reset after a several unsuccessful verification code entries.\\        
                \textit{Argument:} This is to reduce a chance of a successful brute-force attack.\\
                \textit{Dependant flows:} Register via email, login, reset credentials, delete account.

                \item A challenge that the Auth generates MUST rely on a cryptographically secure pseudo-random number 
                    generator (CSPRNG) and MUST NOT be predictable. The challenge SHALL NOT be reused.\\
                \textit{Argument:} If the challenge can be predicted or guessed, an unauthorized party might fetch the
                                attestation or perform an attack from Man-in-the-Middle (MITM) vector of attacks, for
                                instance a relay attack. On the other hand, if the challenge is reused, the system 
                                becomes susceptible to the replay attack. Due to that, the challenge should be 
                                viewed as a nonce.\\
                \textit{Dependant flows:} Register via email, login, reset credentials, delete account.
        \end{enumerate}

    \subsection{Digital signatures}
        \begin{enumerate}
                \item Mobile MUST NOT expose private key directly.\\
                \textit{Argument:} Since the digital signature is a basis of authentication, an utmost respect to 
                                protecting the confidentiality of the private key should be implemented.\\
                \textit{Dependant flows:} Register via email, login, reset credentials, delete account.

                \item Mobile MUST always require a biometric authentication to allow access to digital signing 
                    capability.\\        
                \textit{Argument:} This is to avoid phishing attacks or other illegal attempts to get the user's 
                                digital signature without his/hers explicit approval.\\
                \textit{Dependant flows:} Register via email, login, reset credentials, delete account.

                \item Mobile MUST store generated keys in a trusted execution environment (TEE).\\        
                \textit{Argument:} This is to avoid a root attack or other system-level attacks that target storage 
                                memory or execution environment.\\
                \textit{Dependant flows:} Register via email, login, reset credentials, delete account.

                \item When a user is prompted to authenticate a scanned QR code, the Mobile MUST disclose the domain that 
                    requires the authentication and is located inside the QR code itself.\\
                \textit{Argument:} This is to stop phishing attacks and also provide the feed-forward to the user.\\
                \textit{Dependant flows:} Register via email, login, reset credentials, delete account.
        \end{enumerate}

    \subsection{Attestation}
        \begin{enumerate}
                \item An attestation SHOULD NOT be permanently stored in the Browser or on the Server but rather deleted once
                    its validity has been proved and the process of authentication is finished. It is RECOMMENDED to handle 
                    the attestation with an utmost care for its confidentiality.\\
                \textit{Argument:} Since the attestation proves the holder's identity, an attacker that obtains a valid
                                attestation might pose as the user identified by the attestation.\\
                \textit{Dependant flows:} Register via email, login, reset credentials, delete account.  

                \item An attestation MUST NOT be continuously reused after the login succeeds.\\
                \textit{Argument:} An attestation serves as an initial proof of identity and a continuous exchange of the
                                attestation increases the chances of its hijacking. After the initial authentication, the
                                Server should keep track of the user's identity and his/hers permissions within the system.\\
                \textit{Dependant flows:} Register via email, login, reset credentials, delete account.  

                \item An attestation MUST be checked for the time of issuing and \textit{old} attestations MUST NOT be
                    accepted. It is left up to the implementer to define \textit{old}, but it is RECOMMENDED to
                    decline an attestation that was issued over three minutes ago.\\
                \textit{Argument:} Since the attestation proves the holder's identity, an attacker that obtains a valid
                                attestation might pose as the user identified by the attestation. Given that the login
                                flow is not particularly long or dependant upon data entry, a three minute time-to-live 
                                (TTL) period is gracious enough to let the network exchanges happen.\\
                \textit{Dependant flows:} Register via email, login, reset credentials, delete account. 
        \end{enumerate}
    
    \subsection{Security and privacy concerns}
        \begin{enumerate}
                \item The communication between the Mobile and Auth MUST rely on a secure channel that provides 
                    confidentiality and verifies integrity. It is RECOMMENDED to use up-to-date version of 
                    Transport Level Security (TLS) protocol with a well-tested and computationally secure 
                    cipher suite.\\      
                \textit{Argument:} Mobile and Auth exchange user's personal information and transmitting them over a 
                                public channel would pose a privacy concern, as well as create new vectors of attack, 
                                namely social engineering, phishing, and targeted brute-force attacks. The exchanged 
                                information should remain confidential given that the challenge is used to identify 
                                an authentication session, and later on an attestation is used to authenticate the user.\\
                \textit{Dependant flows:} Register via email, login, reset credentials, delete account. 

                \item User's name, email, public key, and private key MUST be deleted from the Mobile and SHOULD be 
                    deleted from the Auth.\\        
                \textit{Argument:} This is to protect the right to the privacy of the user and their right to control 
                                their personal information.\\
                \textit{Dependant flows:} Register via email, login, reset credentials, delete account.  
        \end{enumerate}