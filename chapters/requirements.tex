\section{Requirements}
In this section, a list of requirements and arguments for their inclusion is given. Requirements are grouped into 
semantical units based on the part of the protocol that they refer to.

    \subsection{Verification codes and challenges}
        \begin{enumerate}
            \item Verification code MUST be at least six characters long, contain a character space of at 
                  least ten characters, and be short lived while it SHALL NOT be predictable.\\    
            \textit{Argument:} Verification code should posses a certain entropy and a level of randomness, 
                               just as well as a limited lifetime in order to reduce  the likelihood of a 
                               successful brute-force attack and Man-in-the-Middle (MITM) attack.

            \item Verification code MUST NOT be revealed to third parties or sent over the network, 
                  except when it's sent to the Email by the Auth.\\        
            \textit{Argument:} Verification code represents a shared secret between the entities and revealing 
                               it to other parties or unnecessarily sending it over a (potentially insecure) 
                               network poses a confidentiality breach risk.

            \item Verification code MUST be reset after a several unsuccessful verification code entries.\\        
            \textit{Argument:} This is to reduce a chance of a successful brute-force attack.

            \item A challenge that the Auth generates MUST rely on a cryptographically secure random number 
                  generator (CSRNG) and MUST NOT be predictable.\\
            \textit{Argument:} If the challenge can be predicted or guessed, an unauthorized party might fetch the
                               attestation or perform an attack from MITM vector of attacks, for instance a relay 
                               attack.

            \item A challenge SHALL NOT be reused.\\
            \textit{Argument:} If the challenge is reused, the system becomes susceptible to the replay attack. 
                               Due to that, the challenge should be viewed as a nonce.
        \end{enumerate}

    \subsection{Digital signatures}
        \begin{enumerate}[resume]
            \item Mobile MUST NOT expose private key directly.\\
            \textit{Argument:} Since the digital signature is the basis of authentication, protecting the 
                               confidentiality of the private key is of utmost importance.                               

            \item Mobile MUST always require a biometric authentication to allow access to digital signing 
                  capability.\\        
            \textit{Argument:} This is to avoid cross-site scripting (XSS) attacks or other illegal attempts to 
                               get the user's digital signature without their explicit approval.

            \item Mobile MUST store generated keys in a trusted execution environment (TEE).\\        
            \textit{Argument:} This is to avoid a root attack or other system-level attacks that target storage 
                               memory or execution environment.
        \end{enumerate}

    \subsection{Attestations}
        \begin{enumerate}[resume]
            \item An attestation MUST be checked for the time of issuing and \textit{old} attestations MUST NOT be
                  accepted. It is left up to the implementer to define \textit{old}, but it is RECOMMENDED to
                  decline an attestation that was issued over three minutes ago.\\
            \textit{Argument:} Since the attestation proves the holder's identity, an attacker that obtains a valid
                               attestation might pose as the user identified by the attestation. Given that the login
                               flow is not dependent upon data entry, a three minute time-to-live (TTL) period is 
                               considerate enough to let the network exchanges happen.

            \item An attestation MUST NOT be continuously reused after the login succeeds.\\
            \textit{Argument:} An attestation serves as an initial proof of identity and a continuous exchange of the
                               attestation increases the chances of its hijacking. After the initial authentication, the
                               Server should keep track of the user's identity and their permissions within the system. 
        \end{enumerate}
    
    \subsection{Infrastructure concerns}
        \begin{enumerate}[resume]
            \item The communication between all entities MUST rely on a secure channel that provides confidentiality and 
                  verifies integrity. It is RECOMMENDED to use up-to-date version of Transport Level Security (TLS) protocol 
                  with a well-tested and computationally secure cipher suite.\\      
            \textit{Argument:} Entities exchange user's personal information and transmitting them over a 
                               public channel would pose a privacy concern, as well as create new vectors of attack, 
                               namely social engineering, phishing, and targeted brute-force attacks. Furthermore, 
                               the challenge is used to identify an authentication session, and later on an attestation 
                               is used to authenticate the user, leading to the possibility of hijacking should the channel
                               remain public.                 
        \end{enumerate}
